\chapter{文献综述}
\label{chap:liter}
\section{背景介绍}
近些年来,随着信息技术以及其他领域的发展,逐渐有两个趋势变得明显,并且将极大的影响人们的生活。
\par
一方面,电子,自动化,计算机等技术的提高,使得无人驾驶技术逐渐受到关注。美国汽车工程师协会奖自动驾驶技术被分为六个层级,从最低级的L0级,即无自动化驾驶,人类全权驾驶,到最高级L5,即完全自动化技术,汽车可以在任何的道路和环境条件下,完成所有的驾驶操作,只在可能的情况下会让人类接管。目前的技术水平已经达到了L3级,即在人类提供一定的应答的情况下驾驶。自动驾驶车辆作为一种新型的交通工具,它可以极大地提高出行的安全性和效率\cite{lee2019autonomous}。但是,目前已有的研究中,考虑自动驾驶车辆情境下的交通行为还较少,对于这样会在很大程度上改变交通出行行为的工具,还需要很多的研究者进行研究。
\par
另一方面,“共享经济”的改变逐渐进入人们的视野。共享单车已经从几年前的不见踪影,到现在的俯拾即是。以ofo,Moobike等为首的一众公司将共享单车推进了人们的生活。Airbnb提供了将人们房子的空闲房间共享的平台,使得“酒店”、“旅馆”等住宿形式之外,衍生出一种更加低成本的,低门槛的、便捷的住宿形式。原本私人的房间也称为陌生人之间可以共享的物品。Uber、Lyft、滴滴等公司推出了出行的共享服务。不同的人可以进行拼车,共用同一辆汽车。这样的经济模式不仅方便了人们的生活,而且带来了许多新兴的经济生产力,实现了多方面的共赢。对于交通领域的学者来说,需要紧跟时代脚步,对新兴出现的ride-sharing模式进行研究,为平台和系统的高效率运行提供支持。
\par
本文即在考虑这样的两种趋势下,希望对城市中出现和可能出现的交通模式,通过算法的求解,找到较为优的解,提高系统的运行效率,减少交通拥堵,减少尾气排放,减少路面上的车。更准确的说,本文想研究的问题,有以下三个:
\begin{enumerate}
\item 在假设全部的车辆都为自动驾驶的情形下,每个选择乘坐自动驾驶车辆出行的乘客都通过中心平台申请自己的出行需求,包括出发的时间,出发的地点,目的地的位置,所能接受的最晚到达时间。我们希望设计一种机制和算法,将所有的乘客出行通过它们的时间和空间关系,串联起来,串联在一起的出行分配给同一辆车接送,我们的目标是希望找到最少的车辆服务所有的出行。这我们将在第~\ref{chap:num}~章进行详细的讨论。
\item 在允许不同的出行之间进行合乘的情形下,每个可以接受合乘的乘客通过中心平台向平台发送出行请求,输入包括出发的时间,出发的地点,目的地的位置,所能接受的到达的最晚时间。我们考虑的合乘为较简单情形的合乘——两个出行进行合乘。我们根据合乘后的出行的数据进行串联,每一条串联的路上的出行都被分配给同一辆车进行接送,我们希望减少路面上的车辆,所以目标为以最少的车辆服务所有的出行需求。这我们将在第~\ref{chap:share}~章进行详细的讨论。
\item 在考虑了两个出行的合乘之后,我们进一步考虑多个出行的合乘,并且考虑车辆的容量,我们希望研究出行之间的组合,即合乘的机制和方法,以及将合乘后的出行分配给车辆的算法。由于问题的难度很大,我们只研究在很短时间内的出行组合以及出行和车辆的匹配。我们通过数学建模,提出解决问题的最优解的算法,并通过实验讨论算法的效率。这一部分我们将在第~\ref{chap:ridesharing}~章详细讨论。
\end{enumerate}

\section{前人研究}
